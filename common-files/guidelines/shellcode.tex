%%%%%%%%%%%%%%%%%%%%%%%%%%%%%%%%%%%%%%%%%%%%%%%%%%%%%%%%
% This file is used by the following labs. Please check 
% these labs if changes are made here, so they don't cause
% problems to those labs:
%     - Buffer_Overflow_Server
%     - Format_String
%%%%%%%%%%%%%%%%%%%%%%%%%%%%%%%%%%%%%%%%%%%%%%%%%%%%%%%%


Shellcode is typically used in code injection attacks.
It is basically a piece of code that launches a shell, and
is usually written in assembly languages.
In this lab, we only provide the binary version of a generic shellcode,
without explaining how it works, because it is non-trivial.
If you are interested in how exactly shellcode works, and
want to write a shellcode from scratch, you
can learn that from a separate SEED lab called \textit{Shellcode Lab}.
Our generic shellcode is listed in the following. We only list
the x86 version (32-bit); the amd64 and arm64 versions of shellcode
are similar. 

\begin{lstlisting}[language=python]
shellcode = (
   "\xeb\x29\x5b\x31\xc0\x88\x43\x09\x88\x43\x0c\x88\x43\x47\x89\x5b"
   "\x48\x8d\x4b\x0a\x89\x4b\x4c\x8d\x4b\x0d\x89\x4b\x50\x89\x43\x54"
   "\x8d\x4b\x48\x31\xd2\x31\xc0\xb0\x0b\xcd\x80\xe8\xd2\xff\xff\xff"
   "/bin/bash*"                                                     (*@\ding{202}@*)
   "-c*"                                                            (*@\ding{203}@*)
   "/bin/ls -l; echo Hello; /bin/tail -n 2 /etc/passwd        *"    (*@\ding{204}@*)
   # The * in this line serves as the position marker         *
   "AAAA"   # Placeholder for argv[0] --> "/bin/bash"
   "BBBB"   # Placeholder for argv[1] --> "-c"
   "CCCC"   # Placeholder for argv[2] --> the command string
   "DDDD"   # Placeholder for argv[3] --> NULL
).encode('latin-1')
\end{lstlisting}

The shellcode runs the \texttt{"/bin/bash"} shell program (Line~\ding{202}),
but it is given two arguments, \texttt{"-c"} (Line~\ding{203}) and
a command string (Line~\ding{204}). This indicates that the shell program
will run the commands in the second argument.
The \texttt{*} at the end of these strings is only a placeholder,
and it will be replaced by
one byte of \texttt{0x00} during the execution of the shellcode. 
Each string needs to have a zero at the end, but we cannot put 
zeros in the shellcode. Instead, we put a placeholder at the end of each string, 
and then dynamically put a zero in the placeholder during the 
execution. 

If we want the shellcode to run some other commands,
we just need to modify the command string in Line~\ding{204}.
However, when making changes, we need to
make sure not to change the length of this string, because the
starting position of the placeholder for the \texttt{argv[]} array,
which is right after the command string,
is hardcoded in the binary portion of the shellcode. If
we change the length, we need to modify the binary part.
To keep the star at the end of this string at the same position,
you can add or delete spaces.


