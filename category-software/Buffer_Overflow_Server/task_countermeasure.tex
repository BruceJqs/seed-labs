
% *******************************************
% SECTION
% *******************************************
\section{Task 6: Experimenting with the Address Randomization}

At the beginning of this lab, we turned off one of the countermeasures,
the Address Space Layout Randomization (ASLR). In this task, we will turn
it back on, and see how it affects the attack. You can run 
the following command on your VM to enable ASLR. This change is global, and 
it will affect all the containers running inside the VM.


\begin{lstlisting}
$ sudo /sbin/sysctl -w kernel.randomize_va_space=2
\end{lstlisting}

Please send a \texttt{hello} message to the Level 1 and Level 3 servers,
and do it multiple times.
In your report, please report your observation, and explain why
ASLR makes the buffer-overflow attack more difficult. 


\ifdefined\arm
\else
% The following activity is only for 32-bit server programs.
\paragraph{Defeating the 32-bit randomization}
It was reported that on 32-bit Linux machines, only 19 bites can be used 
for address randomization.
That is not enough, and we can easily hit the target 
if we run the attack for sufficient number of times. For 64-bit 
machines, the number of bits used for randomization is 
significantly increased. 

In this task, we will give it a try on the 32-bit Level 1 server. 
We use the brute-force approach to attack the server repeatedly, hoping that 
the address we put in our payload can eventually be correct. 
We will use the payload from the Level-1 attack. 
You can use the following shell script to run the vulnerable program in an infinite loop. 
If you get a reverse shell, the script will stop; otherwise, it will keep running. 
If you are not so unlucky, you should be able to get a reverse shell within 10 minutes. 


\begin{lstlisting}[language=bash]
#!/bin/bash

SECONDS=0
value=0
while true; do
  value=$(( $value + 1 ))
  duration=$SECONDS
  min=$(($duration / 60))
  sec=$(($duration % 60))
  echo "$min minutes and $sec seconds elapsed."
  echo "The program has been running $value times so far."
  cat badfile | nc 10.9.0.5 9090
done
\end{lstlisting}
\fi


% *******************************************
% SECTION
% *******************************************
\section{Tasks 7: Experimenting with Other Countermeasures}

% -------------------------------------------
% SUBSECTION
% -------------------------------------------
\subsection{Task 7.a: Turn on the StackGuard Protection}

Many compiler, such as \texttt{gcc}, implements a security mechanism called
\textit{StackGuard} to prevent buffer overflows. In the presence of this
protection, buffer overflow attacks will not work.
The provided vulnerable programs were compiled without 
enabling the StackGuard protection.
In this task, we will turn it on and see what will happen.


Please go to the \texttt{server-code} folder, remove the 
\texttt{-fno-stack-protector} flag from the 
\texttt{gcc} flag, and compile \texttt{stack.c}. 
We will only use \texttt{stack-L1}, but 
instead of running it in a container, we will directly 
run it from the command line. Let's create a file
that can cause buffer overflow, and then feed the 
content of the file \texttt{stack-L1}. Please 
describe and explain your observations. 

\begin{lstlisting}
$ ./stack-L1 < badfile 
\end{lstlisting}
 

% -------------------------------------------
% SUBSECTION
% -------------------------------------------
\subsection{Task 7.b: Turn on the Non-executable Stack Protection}


Operating systems used to allow executable stacks, but
this has now changed: In Ubuntu OS, the binary images of programs (and shared libraries)
must declare whether they require executable stacks or not, i.e., they need to
mark a field in the program header. Kernel or dynamic linker uses this marking
to decide whether to make the stack of this running program executable or
non-executable. This marking is done automatically by the
\texttt{gcc}, which by default makes stack
non-executable. We can specifically make it non-executable
using the \texttt{"-z noexecstack"} flag in the compilation.
In our previous tasks, we used \texttt{"-z execstack"} to
make stacks executable.


In this task, we will make the stack non-executable. We will do
this experiment in the \texttt{shellcode} folder.
The \texttt{call\_shellcode} program puts a copy of shellcode
on the stack, and then executes the code from the stack.
Please recompile \texttt{call\_shellcode.c} without
the \texttt{"-z execstack"} option.
Run the program and describe and explain your observations.


\paragraph{Defeating the non-executable stack countermeasure.}
It should be noted that non-executable stack only makes it
impossible to run shellcode
on the stack, but it does not prevent buffer-overflow attacks,
because there are other ways to run malicious code after exploiting
a buffer-overflow vulnerability. The {\em return-to-libc} attack
is an example. We have designed a separate lab for that
attack. If you are interested, please see our
Return-to-Libc Attack Lab for details.




